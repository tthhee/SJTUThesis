
\chapter{同步性改善全局最优的递归自修复算法}

\section{自修复算法目标}
根据上文所述,当多机器人编队中度较高的机器人出现缺失时,会导致编队同步性下降。假设在图\ref{fig:修复路径对比}中,空缺位置表示出现机器人缺失,记缺失机器人为$R_f$,则缺失机器人的度${R_f}.Deg = 6$。由引理\ref{lem:degree_syn}可知。若用度小于$R_f.Deg$的机器人修复此缺失机器人,则会提高编队的同步性。若最终修复机器人为$R_r$,度为$R_r.Deg$。则$R_f.Deg-R_r.Deg$的值越大,同步性改善效果越好。
%因此,本文目标之一是用编队中度最小的机器人去修复缺失机器人,使得同步性改善效果最优。
\begin{figure}[!htbp]
	\centering
	\includegraphics[width=8cm,height=8cm]{chapter3/figure3-1.png}
	\bicaption[fig:修复路径对比]{不同修复路径对比}{不同修复路径对比}{Fig}{The comparation of different repairing paths.}
\end{figure}

如何找到一条从缺失机器人到度最小机器人的修复路径是本文算法中主要解决的问题。在这里\textbf{修复路径}定义为从缺失机器人邻居中的某个机器人到编队中度最小机器人所经过的机器人序列$\{R_1,R_2,\dots,R_{k-1},R_k\}$。修复路径上的机器人成为修复机器人,其中$R_1$为缺失机器人邻居中的某个机器人,$R_k$为编队中某个度最小的机器人。\textbf{修复路径的长度}定义为修复路径中包含的修复机器人的个数。在图\ref{fig:修复路径对比}中,每一个虚线框中的机器人序列都属于一条修复路径。从图中可以看到不同的修复路径包含的修复机器人个数不同。定义包含修复机器人个数最少的修复路径为\textbf{最优修复路径},图中绿色虚线框内的修复路径即为一条最优修复路径。如果一条修复路径中的修复机器人序列为$\{R_1,R_2,\dots,R_{k-1},R_k\}$,则本文采用的递归自修复过程如下:\\
\indent (1)从缺失机器人的邻居中选择一个修复机器人$R_1$,$R_1$去填补缺失机器人留下的空缺位置,并与缺失机器人邻居重新建立。\\
\indent (2)$R_1$从其邻居中选择一个修复机器人$R_2$用来填补$R_1$移动后留下的空缺位置,并与$R_1$的邻居重新建立连接。\\
\indent (3)重复上述过程,直到编队中的度最小机器人$R_k$填补了$R_{k-1}$留下的空缺位置。\\
第(3)步结束后,整个修复机器人序列的递归自修复过程结束。在这一过程中,修复机器人的空间位置和网络邻居变化如下:\\
空间位置变化:
\begin{equation}
	P_{R_i} = 
	\begin{cases}
		P_{R_f}, & i=1 \\
		P_{R_{i-1}}, & 1 < i \leq k
	\end{cases}
\end{equation}
网络邻居变化:\\
\begin{equation}
	N_e(R_i) = 
	\begin{cases}
		N_e(R_f) - R_i + R_{i+1}, & i = 1 \\
		N_e(R_{i-1}) - R-i + R_{i+1}, & 1<k \leq k
	\end{cases}
\end{equation}

%本文采用上述递归自修复方式实现自修复过程,编队中机器人无法直接确定距离缺失机器人最近的度最小机器人在哪里。所以本文算法另一目标是找到一条最优的修复路径。
综上所述,当多机器人编队中机器人缺失时,本文自修复算法实现目标如下:\\
\indent\indent 1) 修复编队网络拓扑结构。\\
\indent\indent 2) 实现同步性改善全局最优。\\
\indent\indent 3) 自修复过程实现完全分布式,且修复路径最优。\\

本文算法的实现主要基于以下假设:\\
\indent\indent 1) 机器人缺失后,整个编队网络仍然保持连通。\\
\indent\indent 2) 机器人之间的通信速率足够快,从而忽略通信时间消耗。\\
\indent\indent 3) 缺失机器人不影响其他机器人的运动。

\section{梯度的引入与分析}
根据前人的研究,梯度常被用于通过邻居之间的交互估计分布式距离的一种手段\supercite{SciencePaper,nagpal2003organizing,stoy2006using,rubenstein2012kilobot,meng2011autonomous,terada2008automatic},多应用于有限通信范围下的多传感器网络\supercite{nagpal2003organizing}以及多机器人自组织行为\supercite{SciencePaper,stoy2006using}等问题。主要是通过个体间的局部交互来控制整体。文献\parencite{SciencePaper,stoy2006using,rubenstein2012kilobot,terada2008automatic}中,首先定义一个梯度种子节点,由种子节点产生一个初始梯度值,接下来梯度值从种子节点开始扩散,邻居节点接收到扩散的梯度信息后更新自己的梯度信息,一般情况是如果接受到的梯度小于自身已有的梯度值(自身梯度值初始化为一个很大的数),则更新自身梯度值,否则维持自身梯度值不变。接下来再将自身保持的梯度值向邻居扩散,重复此过程,最终整个网络会形成稳定的梯度分布。每个机器人利用自身的梯度信息便可以知道自己与梯度种子节点的距离。图\ref{fig:Gradient_Sample}展示的是文献\parencite{SciencePaper}中的梯度应用示例。在多机器人自组织行为中,预先指定一个梯度种子节点的梯度值以及它的全局坐标,利用梯度扩散与更新过程在网络中形成稳定的梯度分布,在扩散与更新过程中,每个机器人只与它相邻的机器人通信。最终每个机器人根据自身的梯度值以及梯度种子节点的全局坐标估计出自己的全局坐标,从而完成自组织任务。
\begin{figure}[!htbp]
	\centering
	\includegraphics[width=12cm,height=3cm]{chapter3/figure3-2.png}
	\bicaption[fig:Gradient_Sample]{应用梯度实现多机器人自组织行为}{应用梯度实现多机器人自组织行为}{Fig}{Using Gradient to implement multi-robot self-assembly behaviors.}
\end{figure}

为对比有无梯度信息在多机器人编队自修复中的差别,在本团队前期工作中\supercite{张飞2008移动机器人覆盖问题的研究,liu2015gradient,居建军2015多机器人编队自修复算法设计与实现}分别实现了无梯度和有梯度的多机器人编队自修复算法。在文献\parencite{张飞2008移动机器人覆盖问题的研究}中,编队中机器人缺失后仅根据邻居机器人的度的信息选择修复机器人,选取规则为:\\
\indent 1) 如果缺失机器人$R_f$的邻居$R_j \in N_e(R_f)$的度小于$R_f$的度,即$R_j.Deg \leq R_f.Deg$,则机器人$R_j$成为缺失机器人$R_f$的候选修复机器人。所有满足条件的候选修复机器人组成集合:
\begin{equation}
	C(R_f) = \{ R_j | R_j.Deg \leq R_f.Deg, R_j \in N_e{R_f} \} 
\end{equation} 
\indent 2) 在集合$C(R_f)$中选择度最小的机器人作为缺失机器人$R_f$的修复机器人,如果存在多个度最小机器人,则随机选择一个作为修复机器人。

本文中称此算法为随机递归自修复算法。图\ref{fig:Random_self-healing}中展示了在多机器人编队拓扑中出现一个机器人丢失,应用随机自修复算法进行修复的几种情况。
\begin{figure*}[!htbp]
	\centering
	\begin{tabular}{cc}
		\subfigure[]{\includegraphics[width=4cm,height=3.5cm]{chapter3/figure3-3a.png}} & 
		\hspace{2cm}
		\subfigure[]{\includegraphics[width=4cm,height=3.5cm]{chapter3/figure3-3b.png}} \\

		\subfigure[]{\includegraphics[width=4cm,height=3.5cm]{chapter3/figure3-3c.png}} & 
		\hspace{2cm}
		\subfigure[]{\includegraphics[width=4cm,height=3.5cm]{chapter3/figure3-3d.png}} 
	\end{tabular}
	\bicaption[fig:Random_self-healing]{随机递归自修复算法}{随机递归自修复算法,空缺位置表示此处机器人丢失,红色节点表示修复机器人}{Fig}{Random-based recursive self-healing algorithm.}
\end{figure*}

从图中可以看出随机递归自修复算法在实现自修复时选出的修复路径有很大的随机性,且不同的修复路径性能差异较大。造成这种现象的原因主要是在修复机器人选取规则2中,当存在多个满足条件的候选修复机器人时随机选择其中一个,由此对修复路径造成很大的随机性。

为避免在自修复中引入过大的随机性,本研究团队在之前工作中引入梯度,大大降低了随机性,提高了自修复算法的性能。如图\ref{fig:Gradient_self-healing}所示,是在编队网络中引入梯度,形成稳定的梯度分布,算法中与前人不同的是并不事先指定一个梯度种子节点,而是节点根据自己在拓扑网络中邻居数量判断自己是否为梯度种子节点(又称为梯度源节点)。即梯度源节点的定义如下:
\begin{defn}
	\label{defn:gradient_source}
	若编队中机器人$R_i$的度满足:\\
	\begin{equation}
		\forall R_j \in N_e(R_i), R_j.Deg > R_i.Deg
	\end{equation}
	则称机器人$R_i$为梯度源节点。
\end{defn}
修复机器人选取规则为:\\
\indent 1)在缺失机器人$R_f$的邻居中选择度不大于缺失机器人度的机器人作为候选修复机器人。\\
\indent 2)选择候选修复机器人集合中度最小的机器人作为修复机器人。\\
\indent 3)如果存在多个度最小的候选修复机器人,则选择其中梯度值最小的机器人作为修复机器人。\\
\indent 4)如果存在多个度最小且梯度值最小的候选修复机器人,则从中随机选择一个机器人作为修复机器人。\\
图\ref{fig:Gradient_self-healing}是引入梯度后的自修复情况,本文称此算法为基于梯度的递归自修复算法。
\begin{figure*}[!htbp]
	\centering
	\begin{tabular}{cc}
		\subfigure[]{\includegraphics[width=4cm,height=3.5cm]{chapter3/figure3-4a.png}} & 
		\hspace{2cm}
		\subfigure[]{\includegraphics[width=4cm,height=3.5cm]{chapter3/figure3-4b.png}} \\
		
		\subfigure[]{\includegraphics[width=4cm,height=3.5cm]{chapter3/figure3-4c.png}} & 
		\hspace{2cm}
		\subfigure[]{\includegraphics[width=4cm,height=3.5cm]{chapter3/figure3-4d.png}} 
	\end{tabular}
	\bicaption[fig:Gradient_self-healing]{基于梯度的递归自修复算法}{基于梯度的递归自修复算法,颜色的深浅表示梯度值的大小,颜色越深梯度值越小。黑色节点为梯度源节点。空缺位置表示此处机器人丢失,红色节点表示修复机器人}{Fig}{Gradient-based recursive self-healing algorithm, the shade means different gradient value, the deeper the color, the bigger the gradient value, the vacancy means failure robot and red nodes means repairing robots.}
\end{figure*}
从图中可以看到,引入梯度后修复路径的随机性降低很多,且修复路径是一条最短的修复路径。但是这里存在局部极小问题,在图\ref{fig:Gradient_self-healing}(a)中修复机器人序列中最后的梯度源节点机器人的度并不是全局最小。因此根据引理\ref{lem:degree_syn},同步性改善并不是全局最优。

本文借鉴前人的做法,在多机器人编队自修复算法中利用梯度分布式估计机器人与度最小机器人的距离,从而找到最优的修复路径。但为避免局部极小的问题,编队在梯度扩散与更新过程中所传播的信息不只是梯度信息还包括梯度源节点的度的信息,从而将局部度极小节点而非全局度最小节点更新,下文会具体分析这一过程。定义\textbf{梯度源节点的度}为机器人梯度值的扩散起点的梯度源节点的度的值。

\section{同步性改善全局最优自修复算法}

\subsection{算法框架}
总结前人研究的经验,本文设计了一种能够实现同步性改善全局最优的递归自修复算法。本文算法可以在编队中每个机器人个体上独立运行,从而实现完全分布式控制。本文算法主要根据在自修复过程中个体机器人本身在不同情况下所执行的任务的不同分为5种状态,分别为初始状态、梯度源节点状态、非梯度源节点状态、候选修复状态以及修复状态。在多机器人编队开始运行之后,编队中的机器人首先处于初始状态,通过执行梯度源节点生成算法判断自身是否为梯度源节点,若满足满足梯度源节点条件,则状态转移到梯度源节点状态,否则,状态转移到非梯度源节点状态。处于梯度源节点状态和非梯度源节点状态的机器人进行梯度的更新与扩散,最终在编队网络中形成稳定的梯度分布。编队在出现机器人缺失后执行修复的过程是建立在稳定的梯度分布基础之上的。形成稳定的梯度分布之后,当编队中出现机器人缺失时,缺失机器人的邻居检测到邻居中出现机器人缺失,则由当前状态转移到候选修复状态,所有处于候选修复状态下的机器人采用本文提出的竞选机制进行竞选以决出最终的修复机器人。竞选胜出的机器人状态转移到修复状态,处于修复状态的机器人将修复状态进行传递,知道度最小机器人状态转移到修复状态。图\ref{fig:FSM}的FSM图描述了机器人各个状态之间的转换关系。下文会结合编队拓扑给出各个状态更详细的分析。
\begin{figure}[!htbp]
	\centering
	\includegraphics[width=14cm,height=10cm]{chapter3/figure3-5.png}
	\bicaption[fig:FSM]{同步性改善全局最优自修复算法状态转换图}{同步性改善全局最优自修复算法状态转换图}{Fig}{The FSM for the proposed algorithm.}
\end{figure}

\subsection{初始状态}
当编队刚开始运行时,所有机器人最开始都处于初始状态。在这一状态下机器人会执行相应的初始化工作,初始化的工作之一就是将自身的梯度值以及梯度源节点的度的值设置为一个很大的值$N_A$,用$Gra$表示机器人的梯度值,$DS$表示机器人的梯度源节点的度的值,则$Gra=N_A, DS=N_A$。接下来在这一状态下机器人的主要工作是判断自身是否为梯度源节点。根据梯度源节点的定义,判断自身是否为梯度源节点的伪代码如下:\\
\begin{algorithm}
	\caption{判断是否为梯度源节点}
	\label{IsSourceNode}
	\begin{algorithmic}[1]
		\Require 邻居集合$NeighborList[]$
		\Ensure $Ret$,是否为梯度源节点,$Ret \leftarrow true$是梯度源节点,$Ret \leftarrow false$不是梯度源节点
		\Function{IsSourceNode}{$NeighborList[]$}
			\State $Ret \gets false$
			\For {$d_i \ in \ NeighborList[]$}
				\If {$d_i.Deg \leq Deg$}
					\State $Ret \gets false$
				\EndIf
			\EndFor 
			\State \Return{$Ret$}
		\EndFunction
	\end{algorithmic}	
\end{algorithm}
执行上述程序之后,如果判断自身为梯度源节点则将状态转移到梯度源节点状态,如果判断不是梯度源节点,则将状态转移到非梯度源节点状态。图\ref{fig:initial_status}显示了编队拓扑网络中不同机器人节点执行判断是否为梯度源节点的程序后的判断结果。
\begin{figure}[!htbp]
	\centering
	\includegraphics[width=6cm,height=6cm]{chapter3/figure3-6.png}
	\bicaption[fig:initial_status]{梯度源节点的分布}{梯度源节点的分布}{Fig}{The distribution of gradient source node.}
\end{figure}

\subsection{梯度源节点状态与非梯度源节点状态}
处于梯度源节点状态的机器人首先会将自身的梯度值置零,将梯度源节点的度设置成自身的度的大小,即$Gra=0; DS = Deg$。接下来处于梯度源节点状态的机器人会将自身梯度值$Gra$,梯度源节点的度值$DS$封装成一个数据对$P = [Gra,DS]$,然后将此数据对发送给自己的邻居机器人,由此触发整个编队的梯度扩散过程。根据图\ref{fig:initial_status}中的拓扑结构可知,梯度源节点的邻居包含非梯度源节点状态的机器人。又梯度扩散时整个编队拓扑层面的行为,因此梯度扩散过程是由编队中所有机器人参与完成的,梯度源节点触发梯度扩散过程,非梯度源节点传递梯度。因此下面描述的梯度扩散过程是属于梯度源节点状态和非梯度源节点状态下的机器人的共同行为。前文提到过,在梯度源节点生成时会产生局部极小的问题。如图\ref{fig:initial_status}所示,在初始状态下生成的梯度源节点部分度为2,部分度为3,并非全部拥有全局最小度。因此若在当前梯度源节点的分布下形成稳定的梯度分布,则无法保证最终的修复结果达到同步性改善全局最优。这也是文献\parencite{liu2015gradient}中取得效果。这也是本文算法引入梯度源节点的度的原因。本文算法会在梯度扩散过程中根据梯度源节点的度的信息判断当前梯度源节点是否为全局度最小节点,如果不是全局度最小节点,则将其更新为非梯度源节点,状态转移到非梯度源节点状态。经过这样的更新之后,编队中的梯度源节点全部拥有全局最小度,所有节点的梯度值也全部是由全局度最下节点扩散而来。由此可知,在梯度扩散过程中包含两个可能行为,即梯度的更新与梯度源节点的更新,而梯度源节点的更新只可能发生在梯度源节点状态下的机器人。由于这两个行为在梯度扩散过程中同时存在,因此以下梯度扩散公式包含上述两个行为。

梯度扩散公式如下:\\
${ \forall R_j \in N_e(R_i), \delta \geq 0, \Delta \geq 0, \Delta \gg \delta,}$ \\
\begin{equation}\scriptsize
	P_i(t) = 
	\begin{cases}
		\vspace{0.5cm}
		\left[
		\begin{array}{c}
			R_j.Gra(t-1) + \Delta \\
			R_j.DS(t-1)
		\end{array}
		\right], & 
		\begin{array}{l}
			R_i.DS(t-1) > R_j.DS(t-1) \vee \\
			(R_i.DS(t-1) = R_j.DS(t-1) \wedge R_i.Gra(t-1) +\delta > R_j.Gra(t-1) + \Delta)
		\end{array} \\
		
		\left[
		\begin{array}{c}
		R_i.Gra(t-1) + \delta \\
		R_i.DS(t-1)
		\end{array}
		\right], & 
		\begin{array}{l}
			R_i.DS(t-1) < R_j.DS(t-1) \vee \\
			(R_i.DS(t-1) = R_j.DS(t-1) \wedge R_i.Gra(t-1) +\delta \leq R_j.Gra(t-1) + \Delta)
		\end{array}		
	\end{cases} 
\end{equation}
式中:\\
\indent $t$ —— 梯度传播的次数,一个单位代表邻居间传递一次消息。\\
\indent $P_i(t)$ —— $t$次梯度传播后,机器人$R_i$的信息对。\\
\indent $R_j.Gra(t-1)$ —— $t-1$次梯度传播后机器人$R_j$的梯度值。\\
\indent $R_j.DS(t-1)$ —— $t-1$次梯度传播后机器人$R_j$的梯度源节点的度。\\
\indent $R_i.Gra(t-1)$ —— $t-1$次梯度传播后机器人$R_i$的梯度值。\\
\indent $R_i.DS(t-1)$ —— $t-1$次梯度传播后机器人$R_i$的梯度源节点的度。\\
\indent $\delta$ —— 机器人一次梯度传播后梯度增量。\\
\indent $\Delta$ —— 邻居间扩散的梯度增量。

1.梯度更新:\\
\indent 若机器人接收到的信息对中梯度源节点的度值小于自身梯度源节点的度值,则将自身的梯度源节点的度值更新为接收到的梯度源节点的度值,自身梯度值更新为接收到的梯度值加上邻居间的梯度增量。如果接收到的信息对中梯度源节点的度值等于自身梯度源节点的度值,则比较梯度值,若自身梯度值大于信息对中的梯度值加上邻居间扩散的梯度增量,则将自身梯度值更新为信息对中的梯度值与梯度增量的和。

2.梯度源节点更新:\\
\indent 若自身为梯度源节点,且接收到的信息对中梯度源节点的度值小于自身的度值,则将自身状态转移到非梯度源节点状态,并更新相应的梯度源节点的度和梯度值。

梯度扩散算法的伪代码如下:\\
\begin{algorithm}
	\caption{梯度扩散算法}
	\label{algorithm:gradient_diffusion}
	\begin{algorithmic}[1]
		\Require $R_j.Gra \leftarrow$ 邻居机器人的梯度值,$R_j.DS \leftarrow$  邻居机器人的梯度源节点的度
		\Ensure $Gra \leftarrow$ 机器人自身的梯度值,$DS \leftarrow$  机器人自身的梯度源节点的度
		\Function {GradientDiffusion}{$R_j.Gra,R_j.DS$}
			\If {$R_j.DS < DS$ \textbf{or} ($R_j.DS == DS$ \textbf{and} $Gra > R_j.Gra + \Delta$ )}
				\State $DS \gets R_j.DS$
				\State $Gra \gets R_j.Gra + \Delta$
			\Else
				\State $Gra \gets Gra + \delta$
			\EndIf		
			\State \Return {$Gra, DS$}
		\EndFunction
	\end{algorithmic}
\end{algorithm}

\textbf{注1:} \indent 机器人单位时间的梯度增量$\delta$是为修复机器人执行修复之后导致梯度无法更新的情况而设置的。如图\ref{fig:no_update}所示,部分节点上方的括号内标注了该机器人的梯度与梯度源节点的度,图\ref{fig:no_update}(a)中绿色方框是一条修复路径。如图\ref{fig:no_update}(b),当修复完成时,修复机器人的梯度值仍然小于邻居机器人的梯度值,因此根据上述更新条件,则无法实现更新。所以为避免这种情况的发生,在机器人梯度无法得到更新的情况下,机器人梯度会随着时间累加$\delta$,最终使得自身梯度增大,从而可以实现更新。
\begin{figure*}[!htbp]
	\centering
	\subfigure[]{\includegraphics[width=3cm,height=3cm]{chapter3/figure3-7a.png}}
	\hspace{1cm}
	\subfigure[]{\includegraphics[width=3cm,height=3cm]{chapter3/figure3-7b.png}}
	\bicaption[fig:no_update]{修复机器人梯度无法更新情况}{修复机器人梯度无法更新情况}{Fig}{The situation that repairing robots cannot update gradient.}
\end{figure*}

图\ref{fig:gradient_diffusion}描述了梯度扩散的整个过程,包含梯度更新与梯度源节点的更新。节点上方的括号内数字含义同图\ref{fig:no_update}中的含义,即表示(梯度,梯度源节点的度)。节点颜色的深浅也同样显示了梯度值的大小,颜色越深,梯度值越小,黑色节点表示梯度源节点。图\ref{fig:gradient_diffusion}(c)中,红色数字表示经历了梯度源节点的度的更新。
\begin{figure*}[!htbp]
	\centering
	\subfigure[t=0]{\includegraphics[width=4cm,height=4cm]{chapter3/figure3-8a.png}}
	\hspace{0.5cm}
	\subfigure[t=1]{\includegraphics[width=4cm,height=4cm]{chapter3/figure3-8b.png}}
	\hspace{0.5cm}
	\subfigure[t=2]{\includegraphics[width=4cm,height=4cm]{chapter3/figure3-8c.png}}
	\bicaption[fig:gradient_diffusion]{梯度扩散——梯度更新与梯度源节点的更新}{梯度扩散——梯度更新与梯度源节点的更新}{Fig}{Gradient diffusion —— gradient update and gradient source node update.}
\end{figure*}
由上述梯度扩散过程及图\ref{fig:gradient_diffusion}可的如下结论:

\begin{cor}
	\label{cor:sourcegradient_leastdegree}
	当多机器人编队网络形成稳定的梯度分布之后,编队中所有机器人梯度源节点的度都相同且等于编队中全局度最小机器人的度。
\end{cor}

\begin{cor}
	\label{cor:neighbor_gradient}
	当编队网络形成稳定的梯度分布之后,除梯度源节点外,其他机器人邻居中一定含有梯度值小于自身梯度值的机器人。
\end{cor}

\subsection{候选修复状态}
在编队中形成稳定的梯度分布之后,如果出现机器人缺失,则缺失机器人的邻居检测到缺失后会由当前的梯度源节点状态或者非梯度源节点状态转移到候选修复状态,称处于候选修复状态下的机器人为候选修复机器人。候选修复机器人会与其它候选修复机器人进行比较以确定最终的修复机器人。但是由于本文算法是完全分布式实现,机器人只能与其邻居直接通信,因此缺失机器人的邻居之间可能无法直接进行通信,如图\ref{fig:no_update}(a)所示,缺失机器人周围的邻居都会处于候选修复状态,但是如何能在其中选择出一个最佳的修复机器人呢?为解决这一问题,本文采用竞选机制。如果候选修复机器人竞选成功,则由候选修复状态转移到修复状态,如果竞选失败,则由候选修复状态转移回原状态。

\subsubsection{竞选机制}
候选修复机器人没有办法实现任意两两机器人之间的通信,所以每个机器人没有办法一次性获得所有其他候选修复机器人的信息。然而,处于候选修复状态下的机器人相互通信的目的就是要选择出最佳的修复机器人。因此,候选修复机器人没有必要同时获取到其他所有的候选修复机器人的信息,而只要能够得到所有候选修复机器人中最佳修复机器人信息即可。然后通过判断最佳修复机器人是否是自己即可最终确定出唯一的修复机器人。
\begin{defn}
	最佳修复机器人$R_b$为所有处于候选修复状态的机器人之中梯度值最小的修复机器人,即满足:
	\[
		\forall R_i \in N_e(R_f), R_b.Gra \leq R_i.Gra
	\]
\end{defn}
这里仅比较候选修复机器人的梯度值是因为在梯度源节点状态与非梯度源节点状态下编队形成稳定梯度分布之后,所有机器人的梯度源节点的度值相同。

具体竞选机制如下:\\
\indent 1).每个候选修复机器人保存一个当前最佳修复机器人信息$M$,此信息中包含机器人的编号($id$)以及梯度值($Gra$),并在初始时刻将此信息设置为自身的编号和梯度值,同时开启一个计时器$T$。\\
\indent 2).每个机器人将自身保存的最佳修复机器人信息发送给邻居。\\
\indent 3).候选修复机器人如果接收到邻居传来的最佳修复机器人信息$M_r$,则将自身的最佳修复机器人信息与其进行比较,如比较结果满足:
\[
	(M_r.Gra < M.Gra)\ ||\ (M_r.Gra == M.Gra\ \&\& \  M_r.id < M.id) 
\]
则将自身保存的最佳修复机器人信息更新为接收到的信息。\\
\indent 4).重复执行步骤2,3,4直到定时器达到最大竞选时间$T_{max}$。如果此时保存的最佳修复机器人信息中的$id$等于自身$id$,则自己为最终的修复机器人,状态转移到修复状态,否则状态转移回候选修复状态之前的状态。

算法\ref{algorithm:campaign_initial}和\ref{algorithm:campaign}是竞选机制的伪代码,其中算法\ref{algorithm:campaign_initial}是竞选机制的初始化工作,对应竞选机制的步骤1和2。算法\ref{algorithm:campaign}对应竞选机制的步骤3和4。代码中函数$SendToNeighbor(M)$并未给出具体实现,此函数功能是将信息$M$发送给邻居,属于通信函数。
\begin{algorithm}
	\caption{竞选机制初始化}
	\label{algorithm:campaign_initial}
	\begin{algorithmic}[1]
		\Require $M \leftarrow$ 自身保存的最佳修复机器人信息, $T \leftarrow$ 定时器,单位时间表示邻居间一次通信时间
		\Ensure $M \leftarrow$ 自身保存的最佳修复机器人信息, $T \leftarrow$ 定时器,单位时间表示邻居间一次通信时间
		\Function{CampaignInitial}{$M, T$}
			\State $M.Gra \gets Gra$
			\State $M.id \gets id$
			\State $T \gets 0$
			\State $SendToNeighbor(M)$
			\State \Return{$M,T$}
		\EndFunction
	\end{algorithmic}	
\end{algorithm}

\begin{algorithm}
	\caption{竞选机制}
	\label{algorithm:campaign}
	\begin{algorithmic}[1]
		\Require $M_r \leftarrow$ 邻居机器人发来的最佳修复机器人的信息
		\Ensure $IsRepairingRobot \leftarrow$ 是否为修复机器人,$true \leftarrow $是, $ false \leftarrow$ 不是
		\Function {CampaignForRepairingRobot}{$M_r$}
			\State $T \gets T+1$				
			\If {$T < T_{max}$}
				\If {$M_r.Gra < M.Gra$}
				\State $M.Gra \gets M_r.Gra$
				\State $M.id \gets M_r.id$		
			\Else 
				\If {$M_r.Gra == M.Gra \  \&\& \  M_r.id < M.id$}
%			\algstore{CampaignForRepairingRobot}
%	\end{algorithmic}
%\end{algorithm}
%\begin{algorithm}
%	\begin{algorithmic}[1]
%		\algrestore{CampaignForRepairingRobot}
						\State $M.Gra \gets M_r.Gra$
						\State $M.id \gets M.id$
					\EndIf
				\EndIf
				\State $SendToNeighbor(M)$
			\Else 
				\If {$M.id == id$}
					\State $IsRepairingRobot \gets true$
					\State \Return {$IsRepairingRobot$} 
				\EndIf	
			\EndIf			
		\EndFunction		
	\end{algorithmic}
\end{algorithm}

图\ref{fig:campaign}展示了候选修复机器人之间的竞选过程。黄色节点代表机器人处于候选修复状态,红色节点表示机器人处于修复状态。节点内部的数字表示机器人的编号$id$,节点上方的数据对表示竞选过程中邻居间传递的信息,包含梯度值和机器人编号$(Gra,id)$,其中红色数据对表示最佳修复机器人的信息。从图中可以看出经过3倍的邻居通信时间,所有候选修复机器人保存的最佳修复机器人都相同,且最佳修复机器人的$id = 12$,因此$12$号机器人竞选成功,状态转移到修复状态,其他机器人转移回原状态。
\begin{figure*}[!htbp]
	\centering
	\subfigure[$T=0$]{\includegraphics[width=3.5cm,height=3.5cm]{chapter3/figure3-9a.png}}
	\subfigure[$T=1$]{\includegraphics[width=3.5cm,height=3.5cm]{chapter3/figure3-9b.png}}
	\subfigure[$T=2$]{\includegraphics[width=3.5cm,height=3.5cm]{chapter3/figure3-9c.png}}
	\subfigure[$T=3=T_{max}$]{\includegraphics[width=3.5cm,height=3.5cm]{chapter3/figure3-9d.png}}
	\bicaption[fig:campaign]{候选修复机器人竞选过程}{候选修复机器人竞选过程}{Fig}{The campaign process of repairing robots candidates.}
\end{figure*}

\subsubsection{竞选时间}
\begin{lem}
	\label{lem:negotiation_time}
	最大竞选时间$T_{max}$的选择与候选修复机器人的个数$d_f$和邻居之间的通信时间$t_c$有关,满足:
	\[
		T_{max} > d_f t_c/2
	\]
	
	引理\ref{lem:negotiation_time}的证明。
	
	\begin{proof}
		当出现机器人缺失时,最坏情况下所有候选修复状态的机器人形成一个环形,若在六邻居网络中则如图\ref{fig:campaign}所示。根据上述竞选机制可知,最终$id=12$的机器人被选为修复机器人。在到达最大竞选时间之后,所有候选修复机器人记录的最佳修复机器人都为$id=12$的机器人,因此最终的修复机器人信息是从$id=12$的机器人向其他机器人传播。其中$id=16$的机器人距离$id=12$的机器人最远,信息传播时间需要$d_f t_c/2$,而其他机器人接收到信息的时间都要小于$d_f t_c/2$。因此最大竞选时间需满足:$T_{max} > d_f t_c/2$。
	\end{proof}
\end{lem}

\subsection{修复状态}

\subsubsection{修复状态传递}
当编队中出现机器人缺失时,为实现同步性改善全局最优,需要利用编队中度最小的机器人修复缺失机器人。由前文分析可知,一条修复路径终止于全局度最小机器人。因此候选修复机器人通过竞选确定了修复路径中的第一个修复机器人之后,若此修复机器人不是全局度最小机器人,则应该继续在其邻居中选择修复机器人直到全局度最小机器人,从而形成一条修复路径。由修复状态下的机器人继续选择修复机器人的过程称为修复状态传递。被选中的机器人由当前状态转移到修复状态。处于修复状态的机器人在其邻居中选择下一步的修复机器人的规则如下:\\
\indent a)选择邻居中梯度值最小的机器人最为下一步修复机器人。\\
\indent b)若邻居中存在多个梯度值最小的机器人,则选择$id$最小的机器人作为修复机器人。
\indent c)重复步骤a),b)直到度最小机器人为止。

修复状态下修复传递算法的伪码如下:\\
\begin{algorithm}
	\caption{修复状态传递算法}
	\label{algorithm:repairing_status_deliever}
	\begin{algorithmic}[1]
		\Require $Neighbor[] \leftarrow$ 邻居列表, $len \leftarrow$ 列表长度
		\Ensure $NextRepairingRobot \leftarrow$ 下一步修复机器人
		\Function{TransmitRepairingStatus}{$Neighbor[], len$}
			\State $NextRepairingRobot \gets Neighbor[0]$
			\State $i \gets 0$
			\While {$i < len$}
				\If {$Neighbor[i].Gra < NextRepairingRobot.Gra$}
					\State $NextRepairingRobot \gets Neighbor[i]$
				\Else
					\If {$Neighbor[i].Gra = NextRepairingRobot.Gra \ \&\& 
						\ Neighbor[i].id < NextRepairingRobot.id$}
						\State $NextRepairingRobot \gets Neighbor[i]$
					\EndIf
				\EndIf
				\State $i \gets i+1$
			\EndWhile
%					\algstore{TransmitRepairingStatus}
%	\end{algorithmic}
%\end{algorithm}
%\begin{algorithm}
%	\begin{algorithmic}
%				\algrestore{TransmitRepairingStatus}
			\State \Return{$NextRepairingRobot$}
		\EndFunction			
	\end{algorithmic}	
\end{algorithm}
	
\subsubsection{修复运动}
修复状态下的机器人,当其在邻居中选出下一步修复机器人,并将修复信息传递给下一步修复机器人之后便开始运动。由候选修复状态转移到修复状态的机器人填补缺失机器人,由梯度源节点状态或者非梯度源节点状态转移到修复状态的机器人填补上一步修复机器人留下的空缺位置。当修复机器人运动到目标位置后则由修复状态转移到梯度源节点状态或者非梯度源节点状态。

图\ref{fig:repairing_status}展示了修复状态的机器人的行为。图\ref{fig:repairing_status}(a)中修复机器人根据修复机器人选取规则进行下一步修复机器人的选择。将修复信息传递给下一步修复机器人,邻居中的10号机器人接收到修复信息后由梯度源节点状态转移到修复状态,如图\ref{fig:repairing_status}(b)所示。12号修复机器人在指定完下一步修复机器人后即向缺失机器人的位置开始运动,随后12号修复机器人也开始运动,如图\ref{fig:repairing_status}(c)所示。最终,修复机器人运动到目标位置,修复完成状态由修复状态转移到梯度源节点状态或者非梯度源节点状态。

\begin{figure*}[!htbp]
	\centering
	\subfigure[]{\includegraphics[width=3.5cm,height=3.5cm]{chapter3/figure3-10a.png}}
	\subfigure[]{\includegraphics[width=3.5cm,height=3.5cm]{chapter3/figure3-10b.png}}
	\subfigure[]{\includegraphics[width=3.5cm,height=3.5cm]{chapter3/figure3-10c.png}}
	\subfigure[]{\includegraphics[width=3.5cm,height=3.5cm]{chapter3/figure3-10d.png}}
	\bicaption[fig:repairing_status]{修复状态——修复状态传递与修复运动}{修复状态——修复状态传递与修复运动}{Fig}{RepairingStatus——the transmission of repairing status and repairing movement.}
		
\end{figure*}

\section{算法分析}

\subsection{同步性分析}
本文自修复算法的主要目的就是提高编队在机器人缺失后的运动同步性。前文已经指出编队拓扑邻接矩阵的第二大特征值$\lambda_2$可以用来衡量编队的同步性好坏。根据引理\ref{lem:degree_syn},需要用编队中全局度最小的机器人修复缺失机器人才会使得同步性改善全局最优。根据本文自修复算法,对于编队同步性的改善情况可提出以下引理。
\begin{lem}
	\label{lem:syn_analysis}
	当机器人编队网络中出现机器人缺失时,本文算法可以在编队中选出的修复机器人序列中最后一个修复机器人是全局度最小机器人。修复完成后等效成全局度最小机器人修复缺失机器人,以使得同步性改善全局最优。
	
	引理\ref{lem:syn_analysis}的证明。
	
	\begin{proof}
		采用反证法证明。假设编队中出现机器人缺失后,应用本文算法选出的修复机器人序列为$\{R_1,R_2,\dots,R_k\}$。若序列中的最后一步修复机器人$R_k$不是全局度最小的机器人,则$R_k$的梯度值$R_k.Deg > 0$。根据推论\ref{cor:neighbor_gradient}可知,在机器人$R_k$的邻居中仍然存在梯度值小于自身的机器人。由竞选机制中的修复机器人选取规则可知,此机器人会继续在邻居中选择下一步修复机器人,因此$R_k$不是修复机器人序列的最后一步修复机器人。与假设矛盾,所以假设不成立。所以当机器人编队网络中出现机器人缺失时,本文算法可以在编队中选出的修复机器人序列中最后一个修复机器人是全局度最小机器人。修复完成后等效成全局度最小机器人修复缺失机器人,以使得同步性改善全局最优。
	\end{proof}	
\end{lem}

\subsection{修复路径分析}
使得编队同步性改善达到全局最优是本文算法的目标。但是算法执行的效率也是考量算法优劣的重要指标。其中修复路径的长度是影响效率的一个重要因素。在文献\parencite{张飞2008移动机器人覆盖问题的研究}中的随机自修复算法,由于存在过大的随机性,导致最终的修复路径长短不一,随机性很大。因此,算法效率不高。分析随机自修复算法效率不高的原因可以发现,算法不能保证每一步修复机器人的选择都是有效接近度最小机器人的操作。若用梯度的概念来衡量即是每一步的选择不能保证是沿着梯度下降的方向。所以做了很多无用功。文献\parencite{liu2015gradient}中虽然引入了梯度,但是由于梯度分布的原因,无法保证每一条修复路径的选取都能达到同步性改善全局最优,即容易产生局部极小的问题。本文算法在引入梯度的基础上,采用新的梯度扩散机制,使得最终能够实现同步性改善全局最优,且在此基础上实现修复路径最短。因此,给出以下引理。
\begin{lem}
	\label{lem:repairingPath_analysis}
	在编队形成稳定梯度分布的条件下,本文算法可以找到一条最短的实现同步性改善全局最优的修复路径。
	
	引理\ref{lem:repairingPath_analysis}的证明。
	
	\begin{proof}
		利用归纳法证明。如果梯度源节点为缺失机器人,则不需要修复,修复路径长度为$0$。因此满足引理\ref{lem:repairingPath_analysis}。假设编队中某个机器人缺失后本文算法选择的一条修复路径的修复机器人序列为$\{R_1,R_2,\dots,R_k \}$,且序列中$\{R_1,R_2,\dots,R_{k-1} \}$是从$R_1$到$R_{k-1}$的一条最短路径。接下来证明从$R_1$到$R_k$的路径最短。根据假设,机器人$R_{k-1}$不是梯度源节点,又由推论\ref{cor:neighbor_gradient}可知,在其邻居中至少存在一个梯度值小于$R_{k-1}.Deg$的机器人。又由修复状态下选取下一步修复机器人规则可知,修复机器人在邻居中选择梯度值最小的机器人作为下一步修复机器人,因此机器人$R_k$为机器人$R_{k-1}$邻居中与其梯度差最大的机器人。因此$R_1$到$R_k$的路径最短。
		
		综上所述,在编队形成稳定梯度分布的条件下,本文算法可以找到一条最短的实现同步性改善全局最优的修复路径。
	\end{proof}
\end{lem}

\subsection{修复时间分析}
修复时间$t_r$是指从编队中出现机器人缺失到修复完成的时间。根据文献\parencite{张飞2008移动机器人覆盖问题的研究},对于递归自修复过程,修复时间$t_r$可定义为:
\begin{equation}
	t_r = t_d + t_m = \sum_{i+1}^k t_{di} + t_m = t_{d1} + (k-1)t_{dk} + t_m
\end{equation}
式中,\\
\indent $t_d$ —— 选择时间,选择修复机器人所花费的时间和。\\
\indent $t_m$ —— 运动时间,修复机器人移动到目标位置所需的时间。\\
\indent $t_{di}$ —— 第$i$步修复机器人的选择时间。

假设在一条修复路径中存在4个修复机器人,则这4个修复机器人的修复时间示意图如图\ref{fig:repairing_time}所示。
\begin{figure}[!htbp]
	\centering
	\includegraphics[width=8cm,height=4cm]{chapter3/figure3-11.png}
	\bicaption[fig:repairing_time]{修复时间示意图}{修复时间示意图}{Fig}{Diagram of the repairing time.}
\end{figure}
$t_{d1}$为竞选时间,$t_{m_1}$为竞选成功的第一步修复机器人运动到缺失机器人位置的运动时间。$t_{d2}~t_{d4}$分别为处于修复状态的机器人在选择下一步修复机器人的选择时间和消息传递时间之和。$t_{m2}~t_{m4}$为被选中的后续修复机器人在填补上一步修复机器人留下的空缺时的运动时间。为简化问题描述,则假设编队中机器人与邻居之间的距离都相等。所以运动时间$t_{m1} = t_{m2} = t_{m3} = t_{m4}$。有因为修复机器人$t_2 ~ t_4$都是由上一步修复机器人在邻居中根据相同的选取规则指定的,所以选择时间$t_{d2} = t_{d3} = t_{d4}$。而$t_{d1}$为竞选时间,需要更多的机器人参与信息决策与消息传递,因此$t_{d1} > t_{d2} = t_{d3} = t_{d4}$。又因为在本文算法自修复过程中,每一个修复机器人通知完下一步修复机器人即开始运动,因此总的修复时间即为$t_r$。总的修复时间中只包含一倍的运动时间,而相比选择时间,运动时间占据总的修复时间比例要大得多。因此本文算法在修复时间上更具优势。

\section{本章小结}
本章主要就多机器人编队中出现机器人缺失导致同步性下降的问题提出了一种同步性改善全局最优的递归自修复算法。在分析了早期工作,对比编队中有无梯度信息的差别之后,本文引入梯度信息。并通过改进梯度扩散的方式之后,使得同步性改善达到全局最优。利用有限状态机的方式主要通过5种状态描述本文自修复算法,包括初始状态,梯度源节点状态,非梯度源节点状态,候选修复状态和修复状态。针对每一种状态下算法的执行过程给出了详细的分析和介绍。最后从同步性,修复路径,修复时间等几个方面分析了本文算法的优势。总结起来,本章主要贡献如下:\\
\indent 1)提出了新的梯度扩散过程,在梯度扩散过程中,邻居间传递的信息不仅包括梯度信息,还包括梯度源节点的度的信息,从而可以避免局部极小的问题。\\
\indent 2)利用状态机描述算法,方便算法的实现与机器人的控制。\\
\indent 3)对于所有候选修复下的机器人无法直接通信问题,提出了竞选机制,使得在保证完全分布式控制的条件下实现候选修复状态下的修复机器人的选取。\\
\indent 4)从多个指标分析了本文算法的性能。
