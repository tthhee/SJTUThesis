\chapter{研究总结与展望}

\section{本文研究总结}
多机器人编队同步性在机器人编队执行某些任务时是一个重要的考察指标,如协同搬运,环境探测等。编队同步性下降可能会降低任务执行效率,严重的会导致任务失败。而当编队在实际作业过程中,由于复杂的环境或者干扰造成编队中某些机器人的失效不可避免,这种情况就会导致编队的同步性下降。本文针对这种问题提出了一种同步性改善全局最优的完全分布式自修复算法。算法中机器人仅与邻居进行通信从而实现完全分布式控制。引入了梯度,并设计了一套梯度扩散的机制,保证编队中每个机器人的缺失都能找到一条到达度最小机器人的修复路径。引入竞选机制解决缺失机器人邻居无法通信的问题,最终选出的修复机器人进行递归自修复。最后从仿真和实验两面对算法进行了验证,表明算法的可行性、有效性和稳定性。本文具体研究内容如下:\\
\indent 1) 回顾多机器人编队的历史问题与前人工作,总结出前人在多智能体的自修复方面的研究工作大致可分为3类:直接自修复,密度自修复,递归自修复。分析各种自修复算法的优缺点,提出仍然存在的问题。\\
\indent 2) 根据从前人工作中总结出来的问题,分析多机器人编队网络模型与拓扑分析。在机器人个体运动模型与编队网络拓扑模型的基础上总结出编队网络同步性与拓扑结构的关系。根据相应关系给出自修复规则,即引理\ref{lem:degree_syn}。并介绍了本文采用的递归拓扑切换控制。\\
\indent 3) 根据以上自修复规则,提出同步性改善全局最优的完全分布式自修复算法。利用状态机描述了机器人在算法执行过程中的各个状态以及状态之间的转换关系。在初始状态下根据机器人梯度和梯度源节点的度的新设设计了梯度扩散规则,在此梯度扩散规则下形成的稳定梯度分布可以保证梯度源节点全部都是度最小节点,避免了局部极小问题。在候选修复状态下,针对某一缺失机器人的所有候选修复机器人采用竞选机制,保证所有候选修复机器人之间的信息共享,从而解决非邻居之间无法通信的问题。\\
\indent 4) 针对同步性、修复路径、修复时间等指标对算法进行理论分析,为后序的仿真分析提供理论依据。\\
\indent 5) 设计了一套完整的可用于多机器人编队自修复实验的实验平台。实验平台包括定位系统,机器人个体控制与通信系统,远程控制系统等。\\
\indent 6) 未验证算法的可行性与有效性,设计了112个机器人编队自修复仿真。仿真中再现了随机自修复算法与局部最优自修复算法,将本文算法与两种算法在编队同步性改善、修复机器人个数、修复机器人总移动距离等指标上进行对比。仿真对比包括了单个机器人丢失与多个机器人丢失的情况。最后针对修复机器人再丢失的特殊情况,给出本文算法的仿真过程。\\
\indent 7) 在自主设计的自修复实验平台上进行实际实验,实际结果验证了本文算法的可行性,有效性和稳定性。

\section{未来研究展望}